\documentclass[12pt]{article}
\newcommand{\bottomMargin}{2cm}

\newcommand{\header}{
	ДИЗАЙН И АНАЛИЗ НА АЛГОРИТМИ - \\ ПРАКТИКУМ\\
	\vspace{0.1cm}
        Летен семестър, 2025 г., домашно\\
	\vspace{0.1cm}
}
\newcommand{\tl}{$0,2$ сек.}
\newcommand{\ml}{$256$ MB}

\input{../structure/structure.tex}

\begin{document}
\section{Задача H1. Ракети}
В един футуристичен град на име ``ФМИ`` има редица от $N$ небостъргачи, $i$-тия от тях с широчина $w_i$ и височина $h_i$. Те са построени един до друг, без никакви празнини между тях и без да се застъпват.

Вие сте участвате в отряд, който изстрелва ракети срещу града. Отрядът смята да изстреля $M$ на брой ракети, като планът е $i$-тата ракета да прелети на широчина $x_i$ и височина $y_i$ (считаме, че нулевата широчина е тази, в която започва първата сграда).

Напишете програма \textbf{\texttt{rockets}}, която намира колко ракети ще ударят поне един небостъргач.
\subsection{Вход}
От първия ред на стандартния вход се въвеждат две цели числа n и m – брой на небостъргачите и брой на ракетите. От следващите n реда се въвеждат по две цели числа $w_i$ и $h_i$ – широчината и височината на $i$-тия небостъргач. От следващите m реда се въвеждат по две цели числа $x_j$ и $y_j$ – широчина и височина на $j$-тата ракета

\subsection{Изход}
На първия ред на стандартния изход изведете едно цяло число – броя ракети, които ще ударят поне един небостъргач.

\subsection{Ограничения}
\begin{itemize}
	\item $1\leq N, M \leq 300\ 000$
    \item $1\leq h_i, w_i \leq 10^9$
    \item $1\leq x_i, y_i \leq 10^9$
    %\item В ??? процента от тестовете $1\leq N, M \leq 10^3$
\end{itemize}

\subsection{Пример}
\begin{table}[H]
	\begin{tblr}{|X[12,l]|X[8,l]|X[80,j]|}
		\hline
		\textbf{Вход} & \textbf{Изход} & \textbf{Обяснение на примера} \\
		\hline
		\texttt{4 8  \\
2 3 \\
3 6 \\
2 4 \\
4 2 \\
1 2 \\
3 7 \\
4 2 \\
5 8 \\
7 4 \\
9 1 \\
9 5 \\
12 8 \\
}
		& 
		\texttt{4}
		& 
		{На фигурата е показано разположението на небостъргачите и ракетите. \\
        \includegraphics[width = 10cm]{rockets.png}
        } \\
		\hline
	\end{tblr} 
\end{table}
\FloatBarrier

\end{document}
