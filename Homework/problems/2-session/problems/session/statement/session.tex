\documentclass[12pt]{article}
\newcommand{\bottomMargin}{2cm}

\newcommand{\header}{
	ДИЗАЙН И АНАЛИЗ НА АЛГОРИТМИ - \\ ПРАКТИКУМ\\
	\vspace{0.1cm}
        Летен семестър, 2025 г., домашно\\
	\vspace{0.1cm}
}
\newcommand{\tl}{$0,3$ сек.}
\newcommand{\ml}{$256$ MB}

\input{../structure/structure.tex}

\begin{document}
\section{Задача H2. Сесия}
Боби се заел да се подготвя за предстоящата сесия (да, представете си, още отсега). Той ще държи изпит по всеки един от $N$-те предмета (номерирани с числата от $1$ до $N$), които изучава. Боби се стреми най-ниският резултат на някой от $N$-те предмета да е колкото се може по-голям. Уловката е, че е решил да учи само и единствено в учебно време, за да може през останалото да ходи по дискотеки.

В учебната програма на Боби се очертава да има $M$ учебни седмици, във всяка от които има по точно един час за всеки предмет. Във всеки от $(N.M)$-те часове, той трябва да реши следното:

\begin{itemize}
    \item Нека текущия час да е по $i$-тия предмет. Боби може да отиде на лекция и да си повиши знанията по $i$-тия предмет с $a_i$ единици.
    \item Той също може да не слуша професора и да избере някой предмет $j$ ($1 \leq j \leq N$), по който да учи сам на петия етаж и да повиши знанията си по него с $b_j$ единици. Забележете, че е възможно $i=j$.
\end{itemize}

Боби започва с 0 единици знание по всеки един от предметите. Нека след $(N.M)$-те часа, минималният брой единици знание по предмет, който изучава, да е $K$. Колко е най-голямото $K$, което той може да постигне? Напишете програма \textbf{\texttt{session}}, която го намира.

\subsection{Вход}

От първия ред на стандартния вход се въвеждат целите числа $N$ и $M$. От следващия ред се въвеждат $N$ числа, съответно $a_1,a_2,...,a_N$. От последния ред от стандартния вход се въвеждат $N$ числа, съответно $b_1,b_2,...,b_N$.

\subsection{Изход}
На стандартния изход отпечатайте максималното $K$, което Боби може да постигне.

\subsection{Ограничения}
\begin{itemize}
	\item $1\leq N \leq 300\ 000$
        \item $1\leq a_i, b_i, M \leq 10^9$
\end{itemize}


\subsection{Подзадачи}
\begin{table}[H]
	\begin{tblr}{|Q[c,m]|Q[c,m]|Q[c,m]|Q[c,m]|X[c,m]|}
		\hline
		\textbf{Подзадача} & \textbf{Точки}  &
		$N, M$ & 
		\textbf{Други ограничения} \\
		\hline
		$2$ & $10$  & $M = 1$ & $-$ \\ 
		\hline
		$3$ & $25$  & $N.M \leq 3.10^5$ & $a_i=b_i$ \\
		\hline
		$4$ & $27$  & $N.M \leq 3.10^5$ & $-$ \\
		\hline
		$5$ & $22$  & -- & $a_i=b_i$.  \\
            \hline
		$6$ & $16$  & -- & $-$ \\
		\hline
	\end{tblr}
	\caption*{Подзадача означава, че има тестове, носещи съответния брой точки, които спазват специфичните ограничения от подзадачата. Всеки тест се отнася към единствена подзадача}
\end{table}
\FloatBarrier
\pagebreak
\subsection{Пример}
\begin{table}[H]
	\begin{tblr}{|X[12,l]|X[8,l]|X[80,j]|}
		\hline
		\textbf{Вход} & \textbf{Изход} & \textbf{Обяснение на примера} \\
		\hline
		\texttt{3 3\\
19 4 5\\
2 6 2}
		& 
		\texttt{18}
		& 
		{Означението $(x,y) \rightarrow z$ означава „През часа по предмет $y$, през $x$-тата седмица Боби учи за предмет $z$.“, като ако $z=0$, той влиза на съответната лекция, иначе учи сам за предмет $z$.\\
        Едно оптимално учене е: \\
        $(1,1) \rightarrow 2$, $(1,2) \rightarrow 2$, $(1,3) \rightarrow 0$\\
        $(2,1) \rightarrow 0$, $(2,2) \rightarrow 3$, $(2,3) \rightarrow 0$\\
        $(3,1) \rightarrow 3$, $(2,2) \rightarrow 2$, $(3,3) \rightarrow 0$\\
        Единиците знания по предметите са $\{19,18,19\}$. \\
        Така $K=18$, като може да се покаже, че няма по-добро решение.
        } \\
		\hline
            \texttt{2 1\\9 7\\2 6}
		& 
		\texttt{7}
		& 
		{Едно оптимално учене е: \\
        $(1,1) \rightarrow 0$, $(1,2) \rightarrow 0$\\
        Единиците знания по предметите са $\{9,7\}$. \\
        Така $K=7$, като може да се покаже, че няма по-добро решение.} \\
            \hline
            \texttt{4 25\\1 2 3 4\\1 2 3 4}
		& 
		\texttt{48}
		& 
		{Единиците знания по предметите са $\{48,48,48,48\}$. \\
            Така $K=48$, като може да се покаже, че няма по-добро решение.} \\
		\hline
	\end{tblr} 
\end{table}
\FloatBarrier

\end{document}
