\documentclass[12pt]{article}
\newcommand{\bottomMargin}{2cm}

\newcommand{\header}{
	ДИЗАЙН И АНАЛИЗ НА АЛГОРИТМИ - \\ ПРАКТИКУМ\\
	\vspace{0.1cm}
        Летен семестър, 2025 г., първо контролно\\
	\vspace{0.1cm}
}
\newcommand{\tl}{$0,1$ сек.}
\newcommand{\ml}{$256$ MB}

\input{structure/structure.tex}

\begin{document}
\section{Задача К3. Монети}
\begin{wrapfigure}{r}{0.30\textwidth}
	\begin{adjustbox}{width=0.30\textwidth}
		\includegraphics{2025/coins.jpg}
\begin{figure}
	    \centering
	    \includegraphics[width=0.5\linewidth]{coins.jpg}
	    \caption{Enter Caption}
	    \label{fig:enter-label}
	\end{figure}
		\end{adjustbox}
\end{wrapfigure}

След като се прибра от поредната командировка, професор Пикачу започна да разопакова багажа си и откри подарък от скъп приятел от Китай - книгоразделител. Тъй като професорът е неграмотен, единственото приложение на книгоразделителя, което му дойде на ум, беше да го използва като линийка за проектирането на новата си полица за монети. 

Полицата ще представлява множество от $m$ хоризонтално поставени на една височина дървени греди, всяка от които покрива интервала $[L_i, R_i]$. Никои две греди няма да се пресичат или припокриват. Професорът има $n$ монети в своята колекция, но за да ги представи в пълната си прелест, той иска всяка монета да бъде поставена в целочислена координата върху някоя греда, така че минималното разстояние $D$ между две различни монети да бъде колкото може по-голямо. 

Помогнете на професора като напишете програма \textbf{coins.cpp}, която по дадени числа $n$ - брой монети в колекцията, $m$ - брой греди, използвани за изграждането на полицата, и цели числа $L_i, R_i$, $1 \leq i \leq m$ - интервал, който покрива $i$-тата греда, намира максималното възможно $D$, което може да бъде постигнато. 

\subsection{Вход}

От първия ред на стандартния вход се въвеждат две цели положителни числа $n$ - броят
на монетите в колекцията и $m$ - броят греди, използвани за изграждането на полицата.
На всеки от следващите $m$ реда следват по две цели числа - $L_i, R_i$, съответстващи на координатите на левия и десния край на $i$-тата греда. Монета може да бъде поставена в край на интервал.

\subsection{Изход}

На стандартния изход програмата трябва да изведе едно число - максималното разстояние $D$, което може да бъде постигнато.


\subsection{Ограничения}

\vspace{0.1em}
\begin{itemize}
	\item $2 \leq  n \leq 10^5$
    \item $1 \leq  m \leq  10^5$ 
    \item $0 \leq L_i <= R_i \leq 10^{18}$ 
\end{itemize}

\subsection{Оценяване}
В тестове, носещи около $30\%$ от точките, $L_i, R_i \leq 10^5$.

\subsection{Пример}

\begin{table}[ht]
	\begin{tblr}{|l|l|X[j]|}
		\hline
		\textbf{Вход} & \textbf{Изход} & \textbf{Обяснение на примера}\\
		\hline
		\texttt
            {\makecell[lt]{5 3 \\ 9 9 \\ 4 7 \\ 0 2}} & \texttt{2} & Един начин да постигнем разстояние $D = 2$ е да разпределим монетите на координати $0, 2, 4, 6$ и $9$.\\
		\hline
	\end{tblr}
\end{table}
\end{document}
