\documentclass[12pt]{article}
\newcommand{\bottomMargin}{2cm}

\newcommand{\header}{
	ДИЗАЙН И АНАЛИЗ НА АЛГОРИТМИ - \\ ПРАКТИКУМ\\
	\vspace{0.1cm}
        Летен семестър, 2025 г., контролно 1\\
	\vspace{0.1cm}
}
\newcommand{\tl}{$0,1$ сек.}
\newcommand{\ml}{$256$ MB}

\input{structure/structure.tex}

\begin{document}
\section{Задача К2. Ескейп}
\begin{wrapfigure}{r}{0.5\textwidth}
    \centering
    \includegraphics[width = 8.5cm]{structure/escape_image.png}
\end{wrapfigure}
Теодор е решил да пробва най-новата ескейп стая в града. В момента той се намира в помещение с N врати, всяка водеща към различна стая. Във всяка стая има по един контейнер с вода. По-точно, в стая $i$ има контейнер, съдържащ $l_i$ литра вода. До $i$-тия контейнер има туба, чийто обем е $t_i$ литра. Всяка туба е закачена с верига за стената, така че да не може бъде изнасяна от съответната стая. За какво се използва всичко това, така и не става ясно, но това не вълнува особено Теодор в момента. Той вече не мисли за това как да реши пъзела, а как да всее хаос. Неговият план е да покаже на създателите на ескейп стаята защо е лоша идея да слагат вода в нея.

Теодор разполага с $K$ мунити. За всяка минута той отива до някоя стая, в която пълни съответната туба с вода. Не е задължително тубата да бъде напълнена догоре, но е важно да има поне толкова литра останали в контейнера. След това той излива съдържанието й на пода. За да заблуждава гейм-мастърите, които гледат през камерите, Теодор всяка минута се връща до главното помещение.

Помогнете на Теодор като напишете програма \textbf{escape}, която намира колко най-много литра може да излее след $K$ минути.
\subsection{Вход}
От първия ред на стандартния вход се въвеждат две цели числа - съответно $N$ и $K$. От следващите $N$ реда се въвеждат по две цели числа $l_i$ и $t_i$ – съответно обема на контейнера и тубата в $i$-тата стая.

\subsection{Изход}
На първия ред на стандартния изход изведете едно цяло число – максималният брой литри, които Теодор може да излее.

\subsection{Ограничения}
\begin{itemize}
	\item $1\leq N \leq 100\ 000$
    \item $1\leq K \leq 10^{14}$
    \item $1\leq l_i, t_i \leq 10^9$
    %\item В ??? процента от тестовете $1\leq N, M \leq 10^3$
\end{itemize}

\subsection{Пример}
\begin{table}[H]
	\begin{tblr}{|X[12,l]|X[8,l]|X[80,j]|}
		\hline
		\textbf{Вход} & \textbf{Изход} & \textbf{Обяснение на примера} \\
		\hline
		\texttt{3 5 \\
10 4 \\
2 1 \\
13 5 \\
}
		& 
		\texttt{21}
		& 
		{Теодор може да разпредели времето си по следния начин: \\
        $2$ минути за стая $1$ \\
        $3$ минути за стая $3$ \\
        } \\
		\hline
        \texttt{6 21 \\
18 4 \\
30 7 \\
29 5 \\
13 4 \\
3 1 \\
9 3 \\
}
		& 
		\texttt{96}
		& 
		{Теодор може да разпредели времето си така: \\
        $4$ минути за стая $1$ \\
        $5$ минути за стая $2$ \\
        $6$ минути за стая $3$ \\
        $3$ минути за стая $4$ \\
        $3$ минути за стая $6$ \\
        } \\
		\hline
	\end{tblr} 
\end{table}
\FloatBarrier

\end{document}
