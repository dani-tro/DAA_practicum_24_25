\documentclass[12pt]{article}
\newcommand{\bottomMargin}{2cm}

\newcommand{\header}{
	ДИЗАЙН И АНАЛИЗ НА АЛГОРИТМИ - \\ ПРАКТИКУМ\\
	\vspace{0.1cm}
        Летен семестър, 2025 г., първо контролно\\
	\vspace{0.1cm}
}
\newcommand{\tl}{$0,2$ сек.}
\newcommand{\ml}{$256$ MB}

\input{structure/structure.tex}

\begin{document}
\section{Задача К1. Състезание}
\begin{wrapfigure}{r}{0.30\textwidth}
	\begin{adjustbox}{width=0.30\textwidth}
		\includegraphics{2025/pikachu.jpg}
\begin{figure}
	    \centering
	    \includegraphics[width=0.5\linewidth]{pikachu.jpg}
	    \caption{Enter Caption}
	    \label{fig:enter-label}
	\end{figure}
		\end{adjustbox}
\end{wrapfigure}
За участието в предстоящо състезание ФМИ трябва да избере кои от желаещите да участват. Записали се $n$ души, които след провеждане на подборно състезание имали резултати съответно $x_1, x_2, \dots x_n$. Тъй като студентите много се притесняват, че някой от отбора е твърде добър, а такива емоционални вътрешнодушевни тревоги са опасни за представянето на отбора, професор Пикачу иска да избере отбора така, че сбора от резултатите на всеки двама членове на отбора от подборното състезание да бъде по-голям от индивидуалния резултат на всеки един от останалите членове на отбора. Тъй като освен с качество, количественото доминиране също е важен критерий за ФМИ, помогнете на професора, като намерите колко е максималният брой студенти, които могат да образуват отбор, спазвайки гореспоменатите критерии. 

\subsection{Вход}

От първия ред на стандартния вход се въвежда цяло положително число $n$ - броят
на записалите се студенти.
На следващия ред следват $n$ цели числа - $x_1, x_2, \dots x_n$, съответстващи на резултатите на студентите от подборното състезание.

\subsection{Изход}

На стандартния изход програмата трябва да изведе едно число - максималният брой студенти, които могат да образуват отбор.


\subsection{Ограничения}

\vspace{0.1em}
\begin{itemize}
	\item $2 <  n < 200000$
    \item $2879 < x_i < 1000000$ 
\end{itemize}
\subsection{Оценяване}
 В тестове, носещи $50\%$ от точките, $n \leq 20000$. 

\subsection{Пример}

\begin{table}[ht]
	\begin{tblr}{|l|l|X[j]|}
		\hline
		\textbf{Вход} & \textbf{Изход} & \textbf{Обяснение на примера}\\
		\hline
		\texttt
            {\makecell[lt]{5 \\ 4005 5060 10501 3082 8002}} & \texttt{3} & Най-многобройният отбор се състои от първия, втория и четвъртия студент.\\
		\hline
	\end{tblr}
\end{table}
\end{document}
